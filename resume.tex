%%%%%%%%%%%%%%%%%%%%%%%%%%%%%%%%%%%%%%%%%%%%%%%%%%%%%%%%%%%%%%%%%%%%%%
% LaTeX Template: Curriculum Vitae
%
% Source: http://www.howtotex.com/
% Feel free to distribute this template, but please keep the
% referal to HowToTeX.com.
% Date: July 2011
% 
%%%%%%%%%%%%%%%%%%%%%%%%%%%%%%%%%%%%%%%%%%%%%%%%%%%%%%%%%%%%%%%%%%%%%%
% How to use writeLaTeX: 
%
% You edit the source code here on the left, and the preview on the
% right shows you the result within a few seconds.
%
% Bookmark this page and share the URL with your co-authors. They can
% edit at the same time!
%
% You can upload figures, bibliographies, custom classes and
% styles using the files menu.
%
% If you're new to LaTeX, the wikibook is a great place to start:
% http://en.wikibooks.org/wiki/LaTeX
%
%%%%%%%%%%%%%%%%%%%%%%%%%%%%%%%%%%%%%%%%%%%%%%%%%%%%%%%%%%%%%%%%%%%%%%
\documentclass[paper=a4,fontsize=9pt]{scrartcl} % KOMA-article class
							
	\usepackage[english]{babel}
	\usepackage[utf8x]{inputenc}
	\usepackage[protrusion=false,expansion=true]{microtype}
	\usepackage{amsmath,amsfonts,amsthm}     % Math packages
	\usepackage{graphicx}                    % Enable pdflatex
	\usepackage[svgnames]{xcolor}            % Colors by their 'svgnames'
	\usepackage{geometry}
		\textheight=1400px                    % Saving trees ;-)
	\usepackage{url}
	\geometry{
		top=10mm,
	}
	
	\frenchspacing              % Better looking spacings after periods
	\pagestyle{empty}           % No pagenumbers/headers/footers
	
	%%% Custom sectioning (sectsty package)
	%%% ------------------------------------------------------------
	\usepackage{sectsty}
	
	\sectionfont{%			            % Change font of \section command
		\usefont{OT1}{phv}{b}{n}%		% bch-b-n: CharterBT-Bold font
		\sectionrule{0pt}{0pt}{-5pt}{3pt}}
	
	%%% Macros
	%%% ------------------------------------------------------------
	\newlength{\spacebox}
	\settowidth{\spacebox}{8888888888}			% Box to align text
	\newcommand{\sepspace}{\vspace*{0.8em}}		% Vertical space macro
	
	\newcommand{\MyName}[1]{ % Name
			\Huge \usefont{OT1}{phv}{b}{n} \hfill #1
			\par \normalsize \normalfont}
			
	\newcommand{\MySlogan}[1]{ % Slogan (optional)
			\large \usefont{OT1}{phv}{m}{n}\hfill \textit{#1}
			\par \normalsize \normalfont}
	
	\newcommand{\NewPart}[1]{\section*{\uppercase{#1}}}
	
	\newcommand{\PersonalEntry}[2]{
			\noindent\hangindent=2em\hangafter=0 % Indentation
			\parbox{\spacebox}{        % Box to align text
			\textit{#1}}		       % Entry name (birth, address, etc.)
			\hspace{1.5em} #2 \par}    % Entry value

	\newcommand{\PersonalStatement}[2]{
			\noindent\hangindent=0em\hangafter=0 % Indentation
			\parbox{\spacebox}{        % Box to align text
			\textit{#1}}		       % Entry name (birth, address, etc.)
			\hspace{-5.5em} #2 \par}    % Entry value

	\newcommand{\SkillsEntry}[2]{      % Same as \PersonalEntry
			\noindent\hangindent=2em\hangafter=0 % Indentation
			\parbox{\spacebox}{        % Box to align text
			\textit{#1}}			   % Entry name (birth, address, etc.)
			\hspace{1.5em} #2 \par}    % Entry value	
			
	\newcommand{\EducationEntry}[4]{
			\noindent \textbf{#1} \hfill      % Study
			\colorbox{Black}{%
				\parbox{6em}{%
				\hfill\color{White}#2}} \par  % Duration
			\noindent \textit{#3} \par        % School
			\noindent\hangindent=2em\hangafter=0 \small #4 % Description
			\normalsize \par}
	
	\newcommand{\WorkEntry}[4]{				  % Same as \EducationEntry
			\noindent \textbf{#1} \hfill      % Jobname
			\colorbox{White}{\color{Black}#2} \par  % Duration
			\noindent \textit{#3} \par              % Company
			\noindent\hangindent=2em\hangafter=0 \small #4 % Description
			\normalsize \par}
	
	%%% Begin Document
	%%% ------------------------------------------------------------
	\begin{document}
	% you can upload a photo and include it here...
	%\begin{wrapfigure}{l}{0.5\textwidth}
	%	\vspace*{-2em}
	%		\includegraphics[width=0.15\textwidth]{photo}
	%\end{wrapfigure}
	
	\MyName{Brian Fitzgerald}
	
	\sepspace

	\PersonalStatement{}{Software developer passionate about building great user experiences, solving interesting problems, and finding new technologies to learn.}

	
	%%% Personal details
	%%% ------------------------------------------------------------
	\NewPart{Personal details}{}
	
	\PersonalEntry{Location}{Columbia, MO}
	\PersonalEntry{Phone}{(618) 616-3533}
	\PersonalEntry{Mail}{\url{brianfitzgerald242@gmail.com}}
	\PersonalEntry{GitHub}{\url{https://github.com/brianfitzgerald}}
	
	%%% Work experience
	%%% ------------------------------------------------------------
	\NewPart{Work experience}{}

	\WorkEntry{Senior Software Engineer, Robotics}{2021 - 2022}{EquipmentShare}{
		\begin{itemize}
			\item Developed and maintained a distributed simulation environment for testing and developing autonomous construction equipment, including a highly performant LIDAR simulator and test scenario generator.
			\item Worked on various computer vision tasks for an autonomous construction system, including filtering and segmentation of LIDAR and image data.
			\item Developed a pipeline for the recording and replay of sensor data from the robot, including a real-time visualization system for reviewing detections.
			\item Designed and implemented a custom video streaming solution intended for low-bandwidth use, streaming video from the robot's cameras to a remote client, as well as local services.
			\item Assisted a team of interns in developing a custom depth estimation and segmentation system for obstacle detection from video data.
		\end{itemize}
	}
	\sepspace

	\WorkEntry{Senior Software Engineer, Elogs}{2019 - 2021}{EquipmentShare}{
		\begin{itemize}
			\item Worked on expanding support for the ELD compliance system, which gave real-time alerts for FMCSA guideline violations to thousands of drivers.
			\item Used Flask-REST and SQLAlchemy to create a REST API for the ELD compliance system.
			\item Maintained the React Native and TypeScript - based iOS and Android Elogs app, which served a wide range of mobile devices and trackers.
			\item Coordinated with other teams within the company to expand DOT regulation support across multiple products.
			\item Maintained a front-end and API for fleet managers to manage their drivers in React and TypeScript.
			\item Worked to overhaul the backend interview process for the engineering organization.
		\end{itemize}
	}
	\sepspace
	
	\WorkEntry{Software Engineer}{2016 - 2019}{CARFAX}{
		\begin{itemize}
			\item Led the design and implementation of a serverless queue system written in \textbf{Golang} for sending 100,000s of emails per day.
			\item Migrated myCARFAX's on-premises services to an \textbf{AWS}-managed, \textbf{Kubernetes}-based infrastructure.
			\item Assisted with the myCARFAX Shop Data Gateway, a solution for caching millions of items in \textbf{ElastiCache/Redis} and exposing them via a GraphQL API.
			\item Designed and implemented an automated screenshot testing system to identify and alert for visual regressions using \textbf{AWS Lambda} and image processing.
			\item Led the development of the CARFAX Service Shops application, written in \textbf{React / Node / Typescript}, which is used by 1,000s of customers daily.
		\end{itemize}
	}
	\sepspace
	
	\WorkEntry{Developer}{February 2018 - January 2019}{InteraXon Inc}{
		\begin{itemize}
			\item Developed the CES demo experience for the MUSE 2 by Interaxon, an AR experience that allows users to try out a virtual version of the headset and explore its features. Built in \textbf{Unity} with \textbf{ARKit}, utilizing face detection and masking.
			\item Was later deployed in demo kiosks in California, and integrated into InteraXon's marketing material.
		\end{itemize}
	}
	\sepspace


	\WorkEntry{Developer}{September 2018 - February 2019}{Healium XR}{
		\begin{itemize}
			\item Designed and developed several AR experiences for assisting with meditation in \textbf{Unity} with \textbf{ARKit} and \textbf{ARCore}, targeting mobile devices and the Oculus Go headset.
			\item Built a dashboard and analytics platform for tracking biometric data. Used \textbf{AWS Cognito} for authentication and \textbf{Lambda and DynamoDB} for the backend.
		\end{itemize}
	}
	\sepspace

	\WorkEntry{Developer}{June 2017 - May 2018}{GeneTrait Laboratories}{
		\begin{itemize}
			\item Developed the MedTrait reporting system for GeneTrait.
			\item Developed a backend service to generate genetic testing reports with \textbf{Node / Express}, as well as a component library in \textbf{React and Typescript} to render the reports.
			\item Was involved in the entire product development cycle, including designing features and developing a user experience and visual identity.
		\end{itemize}
	}
	\sepspace

	\WorkEntry{Developer}{Oct 2015 - Jun 2016}{QuarkWorks}{
		\begin{itemize}
			\item Worked on a variety of native mobile applications, including ZephyrCharts, an aviation mapping application for pilots. This involved developing \textbf{native iOS and Android clients} for each project.
			\item Developed the companion app for Columbia's local Roots n Blues festival, as well as a backend for managing user data in \textbf{Google Cloud}.
		\end{itemize}
	}
	\sepspace
	
	\WorkEntry{Software Developer}{Feb 2014 - Sep 2015}{WireCloud, LLC}{
		\begin{itemize}
			\item Worked on various contracting projects including the backend for a local food delivery service. Built a system for managing a stream of orders, and assigning them to drivers, as well as a customer-facing order tracking system. 
			\item Developed a companion mobile web application for drivers.
		\end{itemize}
	}
	\sepspace
	
	%%% Skills
	%%% ------------------------------------------------------------
	\NewPart{Skills and Technologies}{}
	
	\SkillsEntry{Frontend}{TypeScript, React, Redux, MobX, Webpack}
	
	\SkillsEntry{Vision}{PCL, Tensorflow, Pytorch, OpenCV, OpenGL, Unity}
	\SkillsEntry{Robotics}{C++, Rust, ZeroMQ, Protobuf, ROS, Waymo Honeycomb}
	\SkillsEntry{Backend}{Express, Golang, Terraform, Kubernetes, Helm, Jenkins, Docker}
	\SkillsEntry{AWS}{SQS, CloudFront / S3, ECS, Lambda, Cognito}

	\SkillsEntry{Mobile}{Swift / iOS, Android SDK, SceneKit}

	\SkillsEntry{XR}{Unity, C\#, ShaderLab, ARKit / ARCore, SteamVR SDK}
	
	\end{document}
	
	
	